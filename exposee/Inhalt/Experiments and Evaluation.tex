\chapter{Experiments and Evaluation}
\label{cha:Experiments and Evaluation}
%\lipsum \autocite{DBLP:books/sp/HarderR01}

\section{Experiments}

The proposed implementation details will be tested in the same scenarios as \autocite{maximilian} and potentially additional ones. \autocite{maximilian} included three parcours with different difficulty levels and conducted 3 different experiments with each trained agent. The first experiment was under optimal conditions with minimal changes from the simulation environment. The second experiment was conducted under different lighting settings. The third one changed the motor power of the agent's two front wheels. Using these settings allows for comparison of the agent developed in this thesis with the previous work and answer the research questions.
In addition, I will also test the agent with different motor power of the two front wheels, this is a more realistic scenario of varying motor power than the one used by \autocite{maximilian}.

\section{Evaluation}

During the training and the final experiments, the agents are evaluated using the success rate, the average time needed to complete the map and the collision rate. The success rate is the percentage of episodes in which the agent successfully completed the map. These metrics were already used by \autocite{maximilian} and measure the most important properties of the agent's behaviour.

There will be additional metrics monitored during the training process to identify weak-points of the agent and erroneous behaviour. Surveilling the training process can provide insights into the agent's behaviour and help with choosing appropriate hyperparameters.
These metrics could be the average cumulative reward, the average number of passed goals, the average distance travelled, the average amount of collisions, the average game duration and the average speed of the agent.



TODO kann ich die Experimente mit varying motor power weglassen?