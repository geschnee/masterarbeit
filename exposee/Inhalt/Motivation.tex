\chapter{Motivation}
%{\let\clearpage\relax \chapter{Motivation}}
\label{cha:Motivation}
%\lipsum \autocite{DBLP:books/sp/HarderR01}

The increasing utilization of artificial intelligence in academia and industry have lead to massive efficiency improvements for all kinds of tasks. The development of autonomous vehicles promisses to greatly reduce the number of traffic accidents and transportation cost. As a result researchers and private enterprises from all over the globe are making progress towards fully autonomous driving agents and integrating them in commercial vehicles, the most well known of these being Tesla FullSelfDriving FSD \autocite{tesla}. Due to the recent developments in artificial intelligence and the very high complexity of the task of autonomous driving, artificial intelligence often plays a big role in these systems \autocite{teslaEndToEnd}.

Predictions for the future of autonomous driving have been very optimistic and although huge progress has been made, the task of autonomous driving is still far from being solved \autocite{state_of_autonomous_driving2023}. This thesis aims to contribute to the research in this field by applying reinforcement learning to autonomous driving agents in a simulated environment. Different implementations will be tested and evaluated to investigate their contributions to the performance of the agent. This work builds upon the work of \autocite{maximilian} and will use the same task and evaluation metrics. %explain why?



% TODO read autonomous driving older paper https://arxiv.org/abs/1710.02410

