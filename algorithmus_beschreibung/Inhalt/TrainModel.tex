\chapter{Motivation}
\label{cha:Motivation}

The increasing utilization of artificial intelligence in academia and industry have lead to massive efficiency improvements for all kinds of tasks. The development of autonomous vehicles promises to greatly reduce the number of traffic accidents and transportation cost \autocite{mckinsey}. As a result, researchers and private enterprises from all over the globe are making progress towards fully autonomous driving agents and integrating them in commercial vehicles, many companies started to integrate adaptive cruise control and lane centering assistance \autocite{carreviews}. Due to the recent developments in artificial intelligence and the very high complexity of the task of autonomous driving, artificial intelligence often plays a big role in these systems \autocite{drl_for_ad}.


Predictions for the future of autonomous driving have been very optimistic and although huge progress has been made, the task of fully autonomous driving is still far from being solved \autocite{state_of_autonomous_driving2023}. This thesis aims at contributing to the research in this field by applying reinforcement learning to autonomous driving agents in a simulated environment. This work builds upon the work of \autocite{maximilian} and will use the same task and evaluation metrics. This thesis focusses on improving the agent's resiliency to changing light conditions by training a convolutional neural network end-to-end using reinforcement learning.