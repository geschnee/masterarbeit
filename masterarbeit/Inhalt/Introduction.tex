\chapter{Introduction}
\label{cha:Introduction}
\acresetall

Recent advancements in \ac{AI} technology have made it possible to develop automated solutions for a wide range of tasks that were previously thought to be too complex and unfit for machines to solve. Most notable in recent few years is the introduction of diffusion image models and large language models. These technologies were well received and moved \ac{AI} tools into public discourse. All over the world people have recognized the potential of \ac{AI} technologies and are now using them in their daily life and at work.

\ac{AI} has already been of great importance in academia and industry for a long time. AI has been proved useful in many different fields, such as image recognition, natural language processing, and robotics. This encourages researchers and industry to further develop and use AI in their work. A promising domain for the application of AI is autonomous driving.

The development of autonomous vehicles promises to greatly reduce the number of traffic accidents and transportation cost \autocite{mckinsey}. The development of autonomous driving could have further downstream effects on our society and industry, such as improved logistic and transportation systems.
As a result, researchers and private enterprises from all over the globe are making progress towards fully autonomous driving agents. Many companies started to integrate adaptive cruise control and lane centering assistance in their products \autocite{carreviews}. Due to the recent developments in \ac{AI} and the very high complexity of the task of autonomous driving, \ac{AI} often plays a big role in these systems \autocite{drl_for_ad}.

Predictions for the future of autonomous driving have been very optimistic and although huge progress has been made, the task of fully autonomous driving is still far from being solved \autocite{state_of_autonomous_driving2023}. This thesis aims at contributing to the research in this field by applying \ac{RL} to autonomous driving agents in a simulated environment. This work builds upon the work of \textcite{maximilian} and will use the same task and evaluation metrics. This thesis focusses on improving the agent's resilience to changing light conditions by training a \ac{CNN} end-to-end using \ac{RL}.

